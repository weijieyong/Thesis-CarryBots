\chapter{結言}
本研究の目的は不安定物体に応じて安定性と機動性を両立した協調運搬可能な群ロボットの開発である.
その前段階として,本論文では,システムの支持部が物体を支えるための条件を検討し,システムのモデリングをした上で,物体が傾けるときの条件を調べた.また,モデリングした式から,ロボットの間の摩擦を上げることで,より安定な支持・運搬作業ができることがわかった.そのほか,それを実現可能な群ロボットシステムを開発した.
そして,実機実験を行い,提案システムの妥当性が確認できた.実験結果より,ロボットが他のロボットに乗り上げられること,移動部が移動するとき不安定な物体を支持できることが確認できた.また,制御することで,より安定な協調運搬も確認できた.
今後の研究としては,まず,移動部のロボット同士をつなげる機構を考え,移動部を紙からロボットに変える.また,移動部と支持部の最適な割合の設計も考えて行きたい.最後に,ロボットの体を生かして,自身で溝を埋めて地面になったり,結合して橋になったりして仲間を渡らせたりすることができる群ロボットシステムにもつなげて行きたい.